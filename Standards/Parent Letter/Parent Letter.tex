\documentclass[14pt, fleqn, paper=letter, oneside]{scrartcl}

\newcommand{\includehead}{false}
\newcommand{\includefoot}{true}

% set basic page format
\usepackage[headsepline=\includehead, footsepline=\includefoot]{scrlayer-scrpage} 
\usepackage[margin=0.5in,
    footskip=1.5\baselineskip,
    headsep=0.5\baselineskip,
    includehead=\includehead, includefoot=\includefoot]{geometry}   %Fixed margins
%\usepackage[compact]{titlesec}

%\usepackage{setspace}
%\onehalfspace
%\doublespace    

% image support
\renewcommand{\topfraction}{0.85}   %Fixes float spacing
\renewcommand{\textfraction}{0.1}
\renewcommand{\floatpagefraction}{0.75}
\usepackage{graphicx}
	\graphicspath{{/Users/fred/Library/TeXShop/Images/}{./Images/}}
\usepackage[space]{grffile}		

% symbol support
\usepackage{siunitx}    %SI unit : \si{\'unit'} or \SI{}{\'unit'}
\sisetup{detect-all}
\usepackage{chemfig}    %Write chemical formulas
\usepackage{mathtools}  %Basic math an extension of amsmath

% formatting support
\usepackage{multicol}   %Multiple cols body : command \multicols{}{'text'}
\usepackage{multirow}   %Multiple row spanning in tables : \multirow{}{width}{'text'}
\usepackage{enumitem}   %Enumeration control
\usepackage{hyperref} % hyperlinks
\usepackage{soul} % highlighting with \hl{command}


% font and date format
\usepackage{newtxtext}
\setkomafont{disposition}{\bfseries}
\usepackage{fancyref}           %Automatically adds Table (tab:' ') or Figure (fig:' ')
%\usepackage{textdate}


\renewcommand{\headfont}{\normalfont}
\renewcommand{\footfont}{\normalfont}

% useful commands
\newcommand{\biu}[1]{\textbf{\emph{\underline{1}}}}
\newcommand{\centerframe}[1]{ % this command makes a box around the content
    {\centering\fbox{\begin{minipage}{0.975\columnwidth}1\end{minipage}}}}

% document commands
%\ihead{Name:}
%\chead{Hour:}
%\ohead{Date:}
\ifoot{Rev. \today}
\cfoot{\maintitle}
\ofoot{\thepage}
\newcommand{\maintitle}{Standards Referenced Grading in Physical Science}

%===================================
\begin{document}
\section*{\maintitle}

\begin{multicols}{2}
For the 2019-2020 school year the Physical Science teachers are switching to standards-referenced grading.
Our goal is to increase clarity when communicating what students actually know or can do.

\subsection*{Background}
There are three pillars to the the science standards passed down to us by the state of Kansas.

\begin{itemize}
  \item Disciplinary Core Ideas
  \item Science and Engineering Practices
  \item Crosscutting Concepts
\end{itemize}

Our standards are based off of those three pillars are.

\begin{itemize}
\item Core Concepts
\item Practices - Lab Notebook
\item Practices - Lab Reports
\end{itemize}

Two of our three standards emphasize the practices of science because these are the tools and skills that will be most universally applicable to students futures.

Throughout both the core ideas and the practices we will incorporate the crosscutting concepts.


\subsection*{Standards Referenced}
A standard is a level of knowledge or demonstration of skill that your student shows.
Rather than work towards and accumulate points, it is the demonstration of a skill or idea that earns a grade.

To emphasize this difference we are shifting from a percentage based system to a numerical scale; 0, 1, 2, 3, and 4.

A 1 means that the student is just developing the concepts.
A 2 means that the student possesses the foundational knowledge required.
A 3 means that students have reached the expected level of understanding.
A 4 means that the student has reached an advanced level of understanding.

In general:
  * 1 is about a D
  * 2 is about a C
  * 3 is about a B
  * 4 is about an A

For your students transcript, we will convert their grades to a typical letter grade.

\subsection*{Our Standards}
\subsubsection*{Core Ideas}
Throughout the year we will have different units.
Each unit will end with a written test.
The average proficiency of ALL of these tests will determine your students grade for this standard.

We believe that traditional testing is an appropriate way of determining conceptual knowledge.

On each test there will be 4 sections corresponding to each of the levels of understanding 1, 2, 3, and 4.
Every student will start with the first section and demonstrate their understanding of those concepts by achieving 90\% or higher on that section.
Students will then work through the subsequently harder sections of the test.

The highest level completed with sufficient mastery (percent scored) will determine the overall standard for that test.

\subsubsection*{Practices - Lab Notebooks}
Your student will keep a lab notebook for the entire year containing all of their notes, experimental observations, journal entries, and classwork throughout the semester.

This standard is a cumulative standard.

Once a week we will sit down with your student and have a conference where we discuss where they are at with the material.
At this time we will review their lab notebook with them, reflecting on the quality of their notes, clarity of writing, observations about experiments, and so on.
Based on this meeting we will assign a score of 1-4 for that week.
The score we assign during the final week will be the final score of that semester.

Students are learning to organize their thoughts and we do not expect them to be perfect from the beginning.
We instead expect students to grow throughout the semester, reaching the expected level by the end.

As such we have two different proficiency scales, one for each semester.
The first semester will focus on completion and clarity, whereas the second semester will focus on how well students are utilizing their notebook as a tool for learning.

\subsubsection*{Practices - Lab Reports}
Each unit will have a single over-arching lab where students will have to develop and test the basic model that the unit is covering.
At the end of the unit, after the test, there will be a work day for the students to finish compiling their observations and notes into a final lab report.

This is essential because communication is a fundamental aspect of ALL endeavors.

This standard is also cumulative.
The students final score will be determined by the last lab report of the semester.


\end{multicols}


































\end{document}